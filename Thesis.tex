\documentclass{mimosis}

\newacronym[description={Principal component analysis}]{PCA}{PCA}{principal component analysis}
\newacronym                                            {SNF}{SNF}{Smith normal form}
\newacronym[description={Topological data analysis}]   {TDA}{TDA}{topological data analysis}

\newglossaryentry{LaTeX}{%
  name        = {\LaTeX},
  description = {A document preparation system},
  sort        = {LaTeX},
}

\newglossaryentry{Real numbers}{%
  name        = {$\real$},
  description = {The set of real numbers},
  sort        = {Real numbers},
}

\makeindex
\makeglossaries

%%%%%%%%%%%%%%%%%%%%%%%%%%%%%%%%%%%%%%%%%%%%%%%%%%%%%%%%%%%%%%%%%%%%%%%%
% Title Page Information
%%%%%%%%%%%%%%%%%%%%%%%%%%%%%%%%%%%%%%%%%%%%%%%%%%%%%%%%%%%%%%%%%%%%%%%%

\title{\textbf{Computational Approaches for Acoustic \& Environmental Informational Utility in Marine Robotics}}
\author{EeShan Chetan Bhatt}
\date{May 4\th 2021}

\begin{document}

\mainmatter

  \def\signature#1#2{\par\noindent#1\dotfill\null\\*
  {\raggedleft #2\par}}

%% --- TITLEPAGE --- %%

% keep this to protect internal macros + use @ as a normal character
\makeatletter

\begin{titlepage}
  \begin{center}
    \begin{Large}
      \@title
    \end{Large}\\[0.1em]
    %
    \emph{\footnotesize by}\\
    {\large \@author} \\[-0.25em]
    B.S.E. YOLO, Mechanical Engineering, \textsc{Muddy Charles University} \\ [2em]
    %
    \begin{singlespace}
    {Submitted to the Joint Program in Oceanography and Applied Ocean Science \& Engineering in partial fulfillment of the requirements for the degree of Doctor of Philosophy in Mechanical \& Oceanographic Engineering} \\
    \end{singlespace}
    %
    \emph{\footnotesize at the}\\
    {\large \textsc{Massachusetts Institute of Technology}} \\
    \emph{\footnotesize and the}\\
    {\large \textsc{Woods Hole Oceanographic Institution}} \\ [2em]
    %
    \begin{singlespace}
    {\copyright1592 E. Beaver. All rights reserved. \\ The author hereby grants to MIT and WHOI permission to reproduce and to distribute publicly copies of this thesis document in whole or in part in any medium now known or hereafter created.} \\ [2em]

    \signature{Author}{\small Department of Mechanical Engineering, MIT \\ Applied Ocean Science \& Engineering, WHOI \\ \@date}
    \vspace{1em}
    \signature{Certified by}{Advisor Name \\ \small Professor of Mechanical and Ocean Engineering, MIT \\ Thesis Supervisor}
    \vspace{1em}
    \signature{Accepted by}{Dept Chair Name \\ \small Professor of Mechanical Engineering, MIT \\ Chair, Department Committee on Graduate Students}
    \vspace{1em}
    \signature{Accepted by}{Dept Chair Name \\ \small Associate Scientist with Tenure, Applied Ocean Physics \& Engineering, WHOI \\ Chair, Joint Committee for Applied Ocean Science \& Engineering}
    \end{singlespace}
  \end{center}
  \makeatother
\end{titlepage}

\newpage
\null
\thispagestyle{empty}
\newpage
  %% ABSTRACT %%
\begin{center}
{\large \@title} \\
\emph{\footnotesize by} \\
\@author \\
\end{center}
%
\noindent \textbf{ABSTRACT} %less than 200 words for WHOI (350 for MIT)

\noindent Enim blandit volutpat maecenas volutpat blandit. Sit amet mattis vulpu
tate enim nulla aliquet porttitor lacus. Purus semper eget duis at tellus at urn
a condimentum mattis. Vitae justo eget magna fermentum iaculis eu. Magnis dis pa
rturient montes nascetur ridiculus mus mauris vitae ultricies. Fringilla phasell
us faucibus scelerisque eleifend donec. Sit amet aliquam id diam maecenas. Ut fa
ucibus pulvinar elementum integer. Suspendisse sed nisi lacus sed viverra tellus
 in hac. Tortor at auctor urna nunc id cursus metus. Semper viverra nam libero j
usto laoreet sit. Dolor sit amet consectetur adipiscing elit. Neque aliquam vest
ibulum morbi blandit cursus. Aliquam sem fringilla ut morbi tincidunt augue inte
rdum velit. Dolor magna eget est lorem ipsum dolor sit amet. Consequat ac felis 
donec et odio pellentesque diam volutpat. Sit amet aliquam id diam maecenas ultr
icies. Viverra mauris in aliquam sem fringilla. Venenatis lectus magna fringilla
 urna porttitor. Risus viverra adipiscing at in tellus integer feugiat scelerisq
ue. \\

\begin{singlespace}
\small{
\noindent Thesis Supervisor: Henrik Schmidt \\
\noindent Title: Professor of Mechanical and Ocean Engineering}
\end{singlespace}

  \tableofcontents

  %%%%%%%%%%%%%%%%%%%%%%%%%%%%%%%%%%%%%%%%%%%%%%%%%%%%%%%%%%%%%%%%%%%%%%%%
\chapter{Introduction}
%%%%%%%%%%%%%%%%%%%%%%%%%%%%%%%%%%%%%%%%%%%%%%%%%%%%%%%%%%%%%%%%%%%%%%%%

\begin{center}
  \begin{minipage}{0.5\textwidth}
    \begin{small}
      In which the reasons for creating this package are laid bare for the
      whole world to see and we encounter some usage guidelines.
    \end{small}
  \end{minipage}
  \vspace{0.5cm}
\end{center}

\noindent This package contains a minimal, modern template for writing your
thesis. While originally meant to be used for a Ph.\,D.\ thesis, you can
equally well use it for your honour thesis, bachelor thesis, and so
on---some adjustments may be necessary, though.

%%%%%%%%%%%%%%%%%%%%%%%%%%%%%%%%%%%%%%%%%%%%%%%%%%%%%%%%%%%%%%%%%%%%%%%%
\section{Why?}
%%%%%%%%%%%%%%%%%%%%%%%%%%%%%%%%%%%%%%%%%%%%%%%%%%%%%%%%%%%%%%%%%%%%%%%%

I was not satisfied with the available templates for \LaTeX{} and wanted
to heed the style advice given by people such as Robert
Bringhurst~\cite{Bringhurst12} or Edward R.\
Tufte~\cite{Tufte90,Tufte01}. While there \emph{are} some packages out
there that attempt to emulate these styles, I found them to be either
too bloated, too playful, or too constraining. This template attempts to
produce a beautiful look without having to resort to any sort of hacks.
I hope you like it.

%%%%%%%%%%%%%%%%%%%%%%%%%%%%%%%%%%%%%%%%%%%%%%%%%%%%%%%%%%%%%%%%%%%%%%%%
\section{How?}
%%%%%%%%%%%%%%%%%%%%%%%%%%%%%%%%%%%%%%%%%%%%%%%%%%%%%%%%%%%%%%%%%%%%%%%%

The package tries to be easy to use. If you are satisfied with the
default settings, just add
%
\begin{verbatim}
\documentclass{mimosis}
\end{verbatim}
%
at the beginning of your document. This is sufficient to use the class.
It is possible to build your document using either \LaTeX|, \XeLaTeX, or
\LuaLaTeX. I personally prefer one of the latter two because they make
it easier to select proper fonts.

%%%%%%%%%%%%%%%%%%%%%%%%%%%%%%%%%%%%%%%%%%%%%%%%%%%%%%%%%%%%%%%%%%%%%%%%
\section{Features}
%%%%%%%%%%%%%%%%%%%%%%%%%%%%%%%%%%%%%%%%%%%%%%%%%%%%%%%%%%%%%%%%%%%%%%%%

%%%%%%%%%%%%%%%%%%%%%%%%%%%%%%%%%%%%%%%%%%%%%%%%%%%%%%%%%%%%%%%%%%%%%%%%
\begin{table}
  \centering
  \begin{tabular}{ll}
    \toprule
    \textbf{Package}      & \textbf{Purpose}\\
    \midrule
      \texttt{amsmath}          & Basic mathematical typography\\
      \texttt{amsthm}           & Basic mathematical environments for proofs etc.\\
      \texttt{booktabs}         & Typographically light rules for tables\\
      \texttt{bookmarks}        & Bookmarks in the resulting PDF\\
      \texttt{dsfont}           & Double-stroke font for mathematical concepts\\
      \texttt{graphicx}         & Graphics\\
      \texttt{hyperref}         & Hyperlinks\\
      \texttt{multirow}         & Permits table content to span multiple rows or columns\\ 
      \texttt{paralist}         & Paragraph~(`in-line') lists and compact enumerations\\
      \texttt{scrlayer-scrpage} & Page headings\\
      \texttt{setspace}         & Line spacing\\
      \texttt{siunitx}          & Proper typesetting of units\\
      \texttt{subcaption} & Proper sub-captions for figures\\
    \bottomrule
  \end{tabular}
  \caption{%
    A list of the most relevant packages required~(and automatically imported) by this template.
  }
  \label{tab:Packages}
\end{table}
%%%%%%%%%%%%%%%%%%%%%%%%%%%%%%%%%%%%%%%%%%%%%%%%%%%%%%%%%%%%%%%%%%%%%%%%

The template automatically imports numerous convenience packages that
aid in your typesetting process. \autoref{tab:Packages} lists the
most important ones. Let's briefly discuss some examples below. Please
refer to the source code for more \index{demonstrations}.

%%%%%%%%%%%%%%%%%%%%%%%%%%%%%%%%%%%%%%%%%%%%%%%%%%%%%%%%%%%%%%%%%%%%%%%%
\subsection{Typesetting mathematics}
%%%%%%%%%%%%%%%%%%%%%%%%%%%%%%%%%%%%%%%%%%%%%%%%%%%%%%%%%%%%%%%%%%%%%%%%

This template uses \verb|amsmath| and \verb|amssymb|, which are the
de-facto standard for typesetting mathematics. Use numbered equations
using the \verb|equation| environment.
%
If you want to show multiple equations and align them, use the
\verb|align| environment:
%
\begin{align}
    V &:= \{ 1, 2, \dots \}\\
    E &:= \big\{ \left(u,v\right) \mid \dist\left(p_u, p_v\right) \leq \epsilon \big\}
\end{align}
%
Define new mathematical operators using \verb|\DeclareMathOperator|.
Some operators are already pre-defined by the template, such as the
distance between two objects. Please see the template for some examples. 
%
Moreover, this template contains a correct differential operator. Use \verb|\diff| to typeset the differential of \index{integrals}:
%
\begin{equation}
  f(u) := \int_{v \in \domain}\dist(u,v)\diff{v}
\end{equation}
%
You can see that, as a courtesy towards most mathematicians, this
template gives you the possibility to refer to the real numbers~$\real$
and the domain~$\domain$ of some function. Take a look at the source for
more examples. By the way, the template comes with spacing fixes for the
automated placement of brackets.

%%%%%%%%%%%%%%%%%%%%%%%%%%%%%%%%%%%%%%%%%%%%%%%%%%%%%%%%%%%%%%%%%%%%%%%%
\subsection{Typesetting text}
%%%%%%%%%%%%%%%%%%%%%%%%%%%%%%%%%%%%%%%%%%%%%%%%%%%%%%%%%%%%%%%%%%%%%%%%

Along with the standard environments, this template offers
\verb|paralist| for lists within paragraphs.
%
Here's a quick example: The American constitution speaks, among others, of
%
\begin{inparaenum}[(i)]
  \item life
  \item liberty
  \item the pursuit of happiness.
\end{inparaenum}
%
These should be added in equal measure to your own conduct. To typeset
units correctly, use the \verb|siunitx| package. For example, you might
want to restrict your daily intake of liberty to \SI{750}{\milli\gram}.

Likewise, as a small pet peeve of mine, I offer specific operators for
\emph{ordinals}. Use \verb|\th| to typeset things like July~4\th
correctly. Or, if you are referring to the 2\nd edition of a book,
please use \verb|\nd|. Likewise, if you came in 3\rd in a marathon, use
\verb|\rd|. This is my 1\st rule.

%%%%%%%%%%%%%%%%%%%%%%%%%%%%%%%%%%%%%%%%%%%%%%%%%%%%%%%%%%%%%%%%%%%%%%%%
\section{Changing things}
%%%%%%%%%%%%%%%%%%%%%%%%%%%%%%%%%%%%%%%%%%%%%%%%%%%%%%%%%%%%%%%%%%%%%%%%

Since this class heavily relies on the \verb|scrbook| class, you can use
\emph{their} styling commands in order to change the look of things. For
example, if you want to change the text in sections to \textbf{bold} you
can just use
%
\begin{verbatim}
  \setkomafont{sectioning}{\normalfont\bfseries}
\end{verbatim}
%
at the end of the document preamble---you don't have to modify the class
file for this. Please consult the source code for more information.


% This ensures that the subsequent sections are being included as root
% items in the bookmark structure of your PDF reader.
\bookmarksetup{startatroot}
\backmatter

  \begingroup
    \let\clearpage\relax
    \glsaddall
    \printglossary[type=\acronymtype]
    \newpage
    \printglossary
  \endgroup

  \printindex
  \printbibliography

\end{document}
